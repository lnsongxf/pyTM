
Dynamic Stackelberg Problems


\section{History dependence}

  Previous chapters described decision problems that are recursive in
what we can call ``natural'' state variables, i.e., state variables that
describe stocks of capital, wealth, and information that helps forecast
future values of prices and quantities that impinge on future utilities
or profits.  In problems that are recursive in the natural state variables,
optimal decision rules are functions of the natural state variables.

\index{time consistency}
 This chapter is our first encounter with a class of
problems that are not recursive in the natural state variables.
Kydland and Prescott (1977), Prescott (1977),  and Calvo (1978) gave
macroeconomic examples of decision problems
whose solutions  exhibited {\it time inconsistency\/}
because they are not recursive in the natural state variables.
Those authors studied the decision problem of  a large agent
(a government) that confronts  a competitive market composed of many small private
agents
whose decisions are influenced by their
{\it forecasts\/} of the government's future actions.  In such
settings, the natural state variables of private agents at time
$t$ are partly shaped by past  decisions
that were influenced by their earlier forecasts of the government's action
at time $t$.  In a rational expectations equilibrium,
 the government  on average confirms private agents' earlier forecasts of
the government's time $t$ actions.  This requirement to confirm prior forecasts
puts constraints on the government's
time $t$ decisions that prevent its problem  from being
recursive in  natural state variables.
These additional constraints make the government's
decision rule at $t$ depend on the entire history of the
state from time $0$ to time $t$.

 \auth{Kydland, Finn E.} \auth{Prescott, Edward C.}%
It took some time for economists to figure out how to formulate policy problems of this type
recursively.
Prescott  (1977) asserted that
 recursive optimal control theory does not apply
to problems with this structure.  This chapter and chapters
\use{socialinsurance} and \use{credible} show
how Prescott's pessimism about the inapplicability
of optimal control theory has been overturned
by more recent work.\NFootnote{The important contribution by Kydland and Prescott (1980)
 helped to dissipate
 Prescott's initial pessimism.}
An important finding is that if the natural state variables
are augmented with  additional state variables that measure
 costs in terms of the government's {\it current\/} continuation
value of confirming {\it past\/} private sector expectations about its
current behavior, this class of problems
can be made recursive. This fact affords immense
computational advantages and yields substantial insights.
  This chapter displays these
within the tractable framework of linear quadratic problems.

\section{The Stackelberg problem}

To exhibit the essential structure of
the problems that concerned Kydland and Prescott (1977) and Calvo (1979),
this chapter uses the optimal linear regulator
to solve a linear quadratic version of
what is known as a dynamic Stackelberg problem.\NFootnote{In some settings
it is  called a Ramsey problem.}
For now we refer to the Stackelberg leader as the government and
the Stackelberg follower as the representative agent or
private sector.  Soon we'll give
an application with another  interpretation of these two players.

Let $z_t$ be  an $n_z \times 1$  vector of natural state variables,
$x_t$ an $n_x \times 1$  vector of  endogenous variables free
to jump at $t$, and $u_t$ a vector of government instruments.
The $z_t$ vector is inherited from the past. But $x_t$ is {\it not\/} inherited from the past.
  The model determines
 the ``jump variables'' $x_t$ at time
$t$.  Included in $x_t$ are prices and quantities that adjust
instantaneously to clear markets at time $t$.
Let $ y_t = \left[\matrix{z_t \cr x_t \cr} \right]$.
Define the
government's one-period loss function\NFootnote{The
 problem assumes that there
are no cross products between states and controls in the
return function.  There is a simple transformation that converts a problem
whose return function has cross products into an equivalent problem
that has no cross products. For example, see Hansen and Sargent (2008, chapter 4, pp. 72-73).}
$$ r(y, u)  =  y' R y  + u' Q u . \EQN target $$

Subject to an initial condition for $z_0$, but not for
$x_0$, a government wants to maximize
$$ -  \sum_{t=0}^\infty \beta^t r(y_t, u_t).  \EQN new1 $$
The government makes policy in light of  the model
$$ \left[\matrix{I & 0 \cr
                 G_{21} & G_{22} \cr}\right]
    \left[ \matrix{   z_{t+1} \cr  x_{t+1} \cr} \right]
  = \left[ \matrix{ \hat A_{11}  &  \hat A_{12} \cr
                    \hat A_{21} & \hat A_{22}  \cr} \right]
\left[\matrix{ z_t \cr x_t \cr} \right]
    + \hat B u_t   . \EQN new2$$
We assume that the matrix on the left is invertible,
so that we can multiply both sides of the above equation by its
inverse to obtain\NFootnote{We
have assumed that the matrix on the left of \Ep{new2} is invertible
for ease of presentation.
However, by appropriately using the
invariant subspace methods described
under step 2 below (see Appendix \the\chapternum\use{appBblkstack}),
it is straightforward to adapt the computational
method when this assumption is violated.}
$$ \left[ \matrix{   z_{t+1} \cr  x_{t+1} \cr} \right]
  = \left[ \matrix{ A_{11}  &   A_{12} \cr
                     A_{21} &  A_{22}  \cr} \right]
   \left[\matrix{ z_t \cr x_t \cr} \right]
    +  B u_t   \EQN new3$$
or
$$ y_{t+1} = A y_t + B u_t  . \EQN new30 $$
The government maximizes \Ep{new1} by choosing sequences
$\{u_t, x_t, z_{t+1}\}_{t=0}^\infty$ subject to
\Ep{new30} and the initial condition for $z_0$.


The private sector's  behavior is summarized by the second
block of equations of \Ep{new2}  or \Ep{new3}. These
typically include the first-order conditions
of private  agents' optimization problem (i.e., their Euler equations). They
summarize the forward-looking aspect of private agents' behavior.
We shall provide an example later in this chapter in which, as is typical
of these problems,
the last $n_x$ equations
of \Ep{new3} or \Ep{new30} constitute {\it implementability constraints} that
 are formed by the Euler equations  of a competitive fringe or
private sector.
When combined with a stability condition to be imposed below, these
Euler equations
summarize the private sector's best response to the
sequence of actions by the government.

The   certainty equivalence principle
stated in chapter \use{dplinear} allows us
to  work with a nonstochastic model.
 We would attain the
same decision rule   if we were  to replace
$x_{t+1}$ with the forecast $E_t x_{t+1}$ and to add  a shock process
$C \epsilon_{t+1}$ to the right side of \Ep{new3}, where
$\epsilon_{t+1}$ is an i.i.d.\ random vector
 with mean of  zero
and identity covariance matrix.

Let $X^t$ denote the history of any variable $X$  from
$0$ to $t$. Miller and Salmon (1982, 1985),
Hansen, Epple, and Roberds (1985), Pearlman, Currie, and Levine (1986),
Sargent (1987),
Pearlman (1992), and others have all studied
versions of the following problem:
\medskip
\noindent {\bf Problem S:}
The  {\it
 Stackelberg problem\/}
is to  maximize \Ep{new1} by choosing an $x_0$ and
 a sequence of decision rules, the time $t$ component of which
maps the time $t$ history of the state
$z^t$ into the time $t$ decision $u_t$ of the Stackelberg leader.
   The Stackelberg leader commits to
this sequence of decision  rules at time $0$.
The maximization is subject to a given initial condition
for $z_0$.  But $x_0$ is among the objects  to be chosen by the Stackelberg leader.

\medskip
The  optimal decision rule  is history dependent, meaning that
$u_t$ depends not only on $z_t$ but also on lags of $z$.
 History
dependence has two sources: (a) the government's
ability to commit\NFootnote{The government would
make different choices were it
to choose sequentially, that is,  were it to select its time $t$ action
at time $t$.}
 to a sequence of rules at time
$0$, and
(b) the forward-looking behavior of the
private sector embedded in the second block of equations
\Ep{new3}.
The history dependence of the
government's plan is expressed in the dynamics of Lagrange multipliers $\mu_x$
on the last $n_x$ equations of  \Ep{new2} or \Ep{new3}.  These
multipliers measure the costs today of honoring past government
promises about current and
future settings
of $u$.  It is appropriate to initialize
the multipliers to zero at time $t=0$, because then there are no past
promises about $u$ to honor.   But the multipliers $\mu_x$
take nonzero values thereafter, reflecting
future costs to the government of adhering to
its commitment.

\section{Solving the Stackelberg problem}
This section describes a remarkable four-step algorithm for
solving the Stackelberg problem.
%Currie and Levine (1996), Pearlman, Currie and  Levine (1986)
%and Pearlman (1992) recognized and exploited the connections
%between the optimal linear regulator problem and
%the optimal control of `forward looking' models
%under commitment. We adapt and extend their arguments
%to include a preference for robustness of decision
%rules to model misspecification.

\subsection{Step 1: solve an optimal linear regulator}

Step 1  seems to disregard the forward-looking aspect  of the
problem (step 3  will take account of that).
%Recall from chapters \use{rgames01} and \use{rgames2t}.
If we temporarily ignore the fact that the $x_0$ component of the state
$y_0 = \left[\matrix{z_0 \cr x_0\cr}\right]$ is {\it not\/}
actually part of the true state vector, then  superficially the Stackelberg
problem
\Ep{new1}, \Ep{new30} has the form of
an optimal linear regulator problem.
It can be solved by forming a Bellman equation and iterating
 until it converges.
The optimal value function has the form $v(y) = - y' P y$, where
$P$ satisfies the  Riccati equation \Ep{bell3}.
The next steps note how the value function
$v(y) = -y'Py$
encodes  objects that solve the Stackelberg problem, then
tell how to decode them.

  A reader not wanting to be reminded
of the details of the Bellman equation can now move directly to step 2.
For those wanting a reminder, here it is.
The linear regulator is
$$ v(y_0) = -y_0' P y_0
= {\rm max}_{\{  u_t, y_{t+1}\}_{t=0}^\infty} - \sum_{t=0}^\infty \beta^t
  \left( y_t' R y_t +   u_t'   Q   u_t \right) \EQN olrp1a $$
where the maximization is subject to a fixed initial condition for
$y_0$ and the law of motion\NFootnote{In step 4, we acknowledge that the $x_0$ component
is {\it not\/} given but is to be chosen by the Stackelberg leader.}
$$ y_{t+1} = A y_t +   B   u_t . \EQN new30a  $$
  Associated with problem \Ep{olrp1a}, \Ep{new30a}  is the
Bellman equation
$$ - y' P y = {\rm max}_{  u, y^*} \left\{ -  y' R y -   u'Q
     u - \beta y^{* \prime} P y^* \right\} \EQN bell1  $$
where the maximization is subject to
$$ y^* = A y + B   u  \EQN bell2 $$
where $y^*$ denotes next period's value of the state.
Problem \Ep{bell1}, \Ep{bell2} gives rise to the matrix Riccati equation
$$ P = R + \beta A' P A - \beta^2 A' P   B (  Q
  + \beta   B' P   B)^{-1}   B' P A  \EQN bell3 $$
and the formula for $F$ in the   decision rule $  u_t = - F y_t$
 $$ F = \beta(   Q + \beta   B' P   B)^{-1}
    B' P A .  \EQN bell4 $$
Thus, we can solve problem \Ep{new1}, \Ep{new30} by iterating
to convergence on the difference equation counterpart to the algebraic Riccati equation \Ep{bell3}, or by using
a faster computational method that emerges as a by-product in
step 2.  This method is described in Appendix \the\chapternum\use{appBblkstack}.


\subsection{Step 2: use the stabilizing properties of shadow price $P y_t$}
 At this point, we decode the information in the matrix
 $P$ in terms of shadow prices that are associated
with  a Lagrangian. We adapt a method described earlier in section \use{lagrangianformulation} that solves a linear quadratic
control problem of the form
%Thus, another way to pose the Stackelberg problem
\Ep{new1}, \Ep{new30} by attaching a sequence of Lagrange
multipliers $ 2 \beta^{t+1} \mu_{t+1}$ to the sequence of constraints
\Ep{new30} and then  forming the Lagrangian:
$$ {\cal L} = - \sum_{t=0}^\infty  \beta^t \left[ y_t' R  y_t + u_t' Q u_t
   + 2  \beta \mu_{t+1}'(A y_t + B u_t  - y_{t+1})
    \right]. \EQN olrp3 $$
For the Stackelberg problem, it is important to partition
$\mu_t$ conformably with our partition of $y_t=\left[\matrix{z_t \cr
   x_t \cr}\right]$, so that
$ \mu_t = \left[\matrix{ \mu_{zt} \cr \mu_{xt} \cr} \right],$
where $\mu_{xt}$ is an $n_x \times 1 $ vector of
 multipliers adhering to the implementability
constraints.  For now, we can ignore the partitioning of
$\mu_t$, but it will be very important when we turn our
attention to the specific requirements of the Stackelberg problem
in step 3.

We want to maximize \Ep{olrp3} with respect to
sequences for $u_t$ and $y_{t+1}$.
The first-order conditions with respect to $u_t, y_t$, respectively,
are:
$$\EQNalign{  0 & = Q u_t + \beta B' \mu_{t+1} \EQN foc1;a  \cr
             \mu_t & = R y_t + \beta A' \mu_{t+1}. \EQN foc1;b \cr} $$
Solving \Ep{foc1;a} for $u_t$
and substituting into \Ep{new30} gives
$$ y_{t+1} = A y_t - \beta B Q^{-1} B'  \mu_{t+1}. \EQN olrp4 $$
%Write \Ep{olrp4} as
%$$ y_{t+1} = A y_t - \beta   B   Q^{-1}   B' \mu_{t+1}.
%      \EQN olrp6$$
We can represent the system formed by
 \Ep{olrp4}  and \Ep{foc1;b}
as
$$ \left[\matrix{I & \beta   B   Q^{-1}   B' \cr
                0 & \beta A' \cr}\right] \left[\matrix{y_{t+1} \cr \mu_{t+1} \cr}
             \right]
 = \left[\matrix{A & 0 \cr
                 - R & I \cr} \right]
  \left[\matrix{y_t \cr \mu_t \cr}\right] \EQN olrp7 $$
or
$$ L^* \left[\matrix{y_{t+1} \cr \mu_{t+1} \cr}
             \right]
 =  N
  \left[\matrix{y_t \cr \mu_t \cr}\right]. \EQN olrp8 $$
We seek a ``stabilizing'' solution of \Ep{olrp8}, i.e., one
that satisfies
$$ \sum_{t=0}^\infty  \beta^t y_t' y_t < +\infty .$$

\subsection{Stabilizing solution}
By the same argument used in  section \use{lagrangianformulation} of chapter \use{dplinear}, a
stabilizing solution satisfies $\mu_0 = P y_0$, where $P$ solves
the matrix Riccati equation \Ep{bell3}. The solution for $\mu_0$
replicates itself over time in the sense that
$$ \mu_t = P y_t . \EQN king4 $$
 Appendix   \the\chapternum\use{appAblkstack} verifies that the matrix $P$ that satisfies the Riccati equation
\Ep{bell3} is the same $P$ that defines the stabilizing initial
conditions $(y_0, P y_0)$.
In Appendix  \the\chapternum\use{appBblkstack}, we describe how to construct $P$ by  computing
generalized eigenvalues and eigenvectors.


\medskip

\subsection{Step 3: convert implementation multipliers into state variables}

\subsubsection{Key insight}
We now confront the fact that the $x_0$ component of $y_0$
consists of variables that are not state variables, i.e., they are
not inherited from the past but are to be determined at time $t$.
In the optimal linear regulator problem, $y_0$ is a state vector
inherited from the past; the multiplier $\mu_0$ jumps at $t$ to
satisfy $\mu_0 = P y_0$ and thereby stabilize the system. But in  the
Stackelberg problem,
 pertinent components of {\it both\/}
$y_0$ {\it and\/} $\mu_0$ must adjust to satisfy
$\mu_0 = P y_0$.
In particular, partition
 $\mu_t$ conformably with the partition of $y_t$  into
$\left[\matrix{z_t' &  x_t' \cr} \right]'$:\NFootnote{This
argument just adapts one in Pearlman (1992). The Lagrangian
associated with the Stackelberg problem remains \Ep{olrp3}, which
means that the same  section \use{lagrangianformulation} logic implies that the stabilizing
solution must satisfy \Ep{king4}. It is only  in how we impose
\Ep{king4} that the solution diverges from that for the linear
regulator.}
$$ \mu_t = \left[\matrix{ \mu_{zt} \cr \mu_{xt} \cr} \right].$$
For the Stackelberg problem,  the first $n_z$ elements
of $y_t$ are predetermined  but the remaining components
are free.  And while the first $n_z$ elements of
$\mu_t$ are free to jump at $t$, the remaining
components are not.  The third step completes the
solution of the Stackelberg problem by acknowledging these facts.
{\it After\/}
we have performed the key step of computing the matrix $P$ that solves
the Riccati equation \Ep{bell3},
we convert the last  $n_x$ Lagrange multipliers $\mu_{xt}$ into
state variables by using the following procedure

%The key insight
%\NFootnote{This
%argument simply adapts the one in Pearlman (1992).}
% is that Lagrangian associated with the
%Stackelberg problem remains \Ep{olrp3} which means that
%the same  logic as in section \use{sect:stabilizing}
%means that the stabilizing solution
%must satisfy
%$$ \mu_t = P y_t . \EQN king4 $$
%It is only  in how we impose \Ep{king4}
%that the solution diverges from that for
%the linear regulator.

Write the last $n_x$ equations
of \Ep{king4} as
$$ \mu_{xt} = P_{21} z_t + P_{22} x_t, \EQN king5 $$
where the partitioning of $P$ is conformable with that
of $y_t$ into $\left[ \matrix{z_t &  x_t \cr}\right]'$.   The vector
$\mu_{xt}$ becomes part of the state at $t$, while $x_t$ is free
to jump at $t$.  Therefore, we solve \Ep{king5}  for
$x_t$ in terms of $(z_t, \mu_{xt})$:
$$ x_t = - P_{22}^{-1} P_{21} z_t + P_{22}^{-1} \mu_{xt}. \EQN king6 $$
Then we can write
$$ y_t =\left[\matrix{z_t \cr x_t \cr}\right]  = \left[ \matrix{I & 0 \cr
           - P_{22}^{-1} P_{21} &  P_{22}^{-1} \cr}  \right]
    \left[\matrix{z_t \cr \mu_{xt} \cr}\right] \EQN king7 $$
and from \Ep{king5}
$$ \mu_{xt} =  \left[\matrix{ P_{21} & P_{22} \cr}
\right] y_t . \EQN king8 $$

With these modifications, the key formulas
\Ep{bell4} and \Ep{bell3} from the optimal linear
regulator for $F$ and $P$, respectively,
continue to apply.
Using \Ep{king7}, the  optimal decision rule is
$$u_t    = -F
 \left[ \matrix{I & 0 \cr
           - P_{22}^{-1} P_{21} &  P_{22}^{-1} \cr}  \right]
    \left[\matrix{z_t \cr \mu_{xt} \cr}\right]. \EQN king10 $$
Then we have  the following complete description of the
Stackelberg plan:\NFootnote{When a random shock $C \epsilon_{t+1}$ is present, we must add
  $$\left[ \matrix{I & 0 \cr P_{21} & P_{22}\cr} \right] C \epsilon_{t+1}
    \EQN randterm $$
to the right side of  \Ep{king11;a}.} {\ninepoint
$$ \EQNalign{ \left[ \matrix{ z_{t+1} \cr \mu_{x,t+1} \cr} \right]
  & = \left[ \matrix{I & 0 \cr P_{21} & P_{22}\cr} \right]
  (A - B F)
  % \left(\matrix{A_1 - B_1 F_1  \cr
   %             A_2 - B_2 F_1 - C_2 F_2 \cr} \right)
    \left[\matrix{ I & 0 \cr
              - P_{22}^{-1} P_{21} & P_{22}^{-1} \cr}\right]
    \left[\matrix{ z_t \cr \mu_{xt} \cr} \right] \hskip1cm \EQN king11;a \cr
   x_t & = \left[\matrix{ - P_{22}^{-1} P_{21}   & P_{22}^{-1} \cr}
  \right]
    \left[\matrix{ z_t \cr \mu_{xt} \cr} \right] .  \EQN king11;b \cr} $$
}%endninepoint
The difference equation
\Ep{king11;a} is to  be initialized from the given
value of $z_0$ and a value for $\mu_{x0}$ to be determined in step 4.
%{\bf XXXX Tom:  check these equations in the monograph on robustness.  In particular,
% check that we zero out \mu_{0,x} and not \mu_{0,z}.}



\subsection{Step 4: solve for $x_0$ and $\mu_{x0}$}

The value function $V(y_0)$ satisfies
$$ V(y_0) = - z_0 ' P_{11} z_0 - 2 x_0' P_{21} z_0 - x_0' P_{22} x_0 . \EQN valuefny $$
Now choose $x_0$ by equating to zero the gradient of $V(y_0)$ with respect to  $x_0$:
$$ - 2 P_{21} z_0 - 2 P_{22} x_0 =0,  $$
which by virtue of \Ep{king5} is equivalent with
$$ \mu_{x0} = 0 . \EQN mu0condition  $$
Then we can compute $x_0$ from \Ep{king6} to arrive at
$$ x_0 = - P_{22}^{-1} P_{21} z_0. \EQN king6x0 $$

The Lagrange multiplier $\mu_{xt}$ measures the cost to the Stackelberg leader at $t \geq 0$
of confirming expectations about its time $t$ action that the followers had held
at dates $s < t$.  
Setting
$\mu_{x0}=0$ means that at time $0$ there are no such prior expectations to confirm.
\subsection{Summary}
In summary, we solve the Stackelberg problem by
formulating a particular optimal linear regulator, solving the associated
matrix Riccati equation \Ep{bell3} for $P$, computing
$F$, and  then partitioning
$P$ to obtain representation \Ep{king11}.




\subsection{History-dependent representation of decision rule}

\auth{Von Zur Muehlen, Peter}%
 For some purposes, it is useful to eliminate the implementation
multipliers  $\mu_{xt}$ and to express
the decision rule for $u_t$ as a function
of $z_t, z_{t-1},$ and $u_{t-1}$.
This can be accomplished as
follows.\NFootnote{Peter Von Zur Muehlen suggested
this representation to us.}
First represent \Ep{king11;a} compactly as
$$  \left[ \matrix{ z_{t+1} \cr \mu_{x,t+1} \cr} \right]
   = \left[ \matrix{m_{11} & m_{12} \cr m_{21} & m_{22}\cr} \right]
    \left[\matrix{ z_t \cr \mu_{xt} \cr} \right] \EQN vonzer1 $$
and write the feedback rule for $u_t$
$$u_t  = f_{11}  z_{t} + f_{12} \mu_{xt} . \EQN vonzer2 $$
Then where $f_{12}^{-1}$ denotes
the generalized inverse of $f_{12}$,
 \Ep{vonzer2} implies $\mu_{x,t} = f_{12}^{-1}(u_t - f_{11}z_t)$.
Equate the right side of this expression to the right side
of the second line of \Ep{vonzer1} lagged once and rearrange by using
\Ep{vonzer2} lagged once to eliminate $\mu_{x,t-1}$
to get
$$ u_t =  f_{12} m_{22} f_{12}^{-1} u_{t-1} + f_{11} z_t
   + f_{12}(m_{21} - m_{22} f_{12}^{-1} f_{11}) z_{t-1}
   \EQN vonzer3;a $$
or
$$ u_t = \rho u_{t-1} + \alpha_0 z_t + \alpha_1 z_{t-1} \EQN vonzer3;b $$
for $t \geq 1$.  For $t=0$, the initialization $\mu_{x,0}=0$ implies
that
$$ u_0 = f_{11} z_0. \EQN vonzer3;c $$

  By making the instrument feed back on itself,
the form of \Ep{vonzer3} potentially allows for
``instrument-smoothing'' to emerge as an optimal rule under
commitment.\NFootnote{This insight partly motivated Woodford
(2003) to use his model to interpret empirical evidence about
interest rate smoothing in the United States.}


\subsection{Digression on determinacy of equilibrium}
Appendix \the\chapternum\use{appBblkstack} describes methods for  solving a system of difference
equations of the form \Ep{new2} or \Ep{new3} with an arbitrary feedback rule that
expresses the decision rule for
$u_t$ as a function of current and  previous values of $y_t$ and perhaps previous values
of itself.  The difference equation system has  a unique solution
satisfying the stability condition $\sum_{t=0}^\infty \beta^t y_t \cdot  y_t$
if the eigenvalues of the matrix \Ep{symplec2} split, with half being greater than
unity and half being less than unity in modulus.  If more than half are less than
unity in modulus, the equilibrium is said to be indeterminate \index{indeterminacy!of equilibrium}%
 in the sense that there are multiple equilibria starting from any initial condition.

If we choose to represent the solution of a Stackelberg or Ramsey problem in the form
\Ep{vonzer3}, we can substitute that representation for $u_t$ into
\Ep{new3}, obtain a difference equation system in $y_t, u_t$, and ask whether
the resulting system is determinate. To answer this question, we would use the method
of Appendix \the\chapternum\use{appBblkstack}, form system \Ep{symplec2}, then check whether the generalized
eigenvalues split as required.   Researchers have used this method to study the determinacy
of equilibria under Stackelberg plans with representations like \Ep{vonzer3} and have discovered
that sometimes an equilibrium can be indeterminate.\NFootnote{The existence of a Stackelberg plan is not at
issue because we know how to construct one using the method in the text.}  See Evans
and Honkapohja (2003) for a discussion of determinacy of equilibria under commitment
in a class of equilibrium monetary models and
how determinacy depends on how the decision rule of the Stackelberg leader is represented.  Evans and
Honkapohja argue that casting a government decision rule in a way that leads to indeterminacy is a bad idea.
\auth{Evans, George W.}
\auth{Honkapohja, Seppo}
\index{implementation!of Stackelberg plan}
\section{A large firm with a competitive fringe}

As an example, this section studies the equilibrium of an industry with
a large firm that acts as a Stackelberg leader with respect to a competitive
fringe.  Sometimes the large firm is called `the monopolist' even though there are
actually many firms in the industry.  The industry produces a single nonstorable homogeneous good. One
large firm produces $Q_t$ and a representative firm in a competitive
fringe produces $q_t$.  The representative firm in the competitive
fringe acts as a price taker and chooses sequentially.  The large firm
commits to a policy at time $0$, taking into account its ability to
manipulate the price sequence, both directly through the effects of
its quantity choices on prices, and indirectly through the responses of
the competitive fringe to its forecasts of prices.\NFootnote{Hansen and Sargent
(2008, ch.~16) use this model as a laboratory to illustrate an equilibrium concept featuring
robustness in which at least one of the agents has doubts about the stochastic specification
of the demand shock process.}

  The costs of production are
${\cal C}_t = e Q_t + .5 g Q_t^2+ .5 c (Q_{t+1} - Q_{t})^2 $
for the large firm
and $ \sigma_t= d q_t + .5 h q_t^2 + .5 c (q_{t+1} - q_t)^2$
for the competitive firm,
where $d>0, e >0, c>0, g >0, h>0 $ are cost parameters.
There is a linear inverse demand curve
$$ p_t = A_0 - A_1 (Q_t + \overline q_t) + v_t, \EQN oli1 $$
where $A_0, A_1$ are both positive and  $v_t$ is a disturbance
to demand governed by
$$ v_{t+1}= \rho v_t + C_\epsilon \check \epsilon_{t+1} \EQN oli2 $$
and where $ | \rho | < 1$ and $\check \epsilon_{t+1}$ is an i.i.d.\
sequence of random variables with mean zero and variance $1$.
In \Ep{oli1}, $\overline q_t$ is equilibrium output  of the representative
competitive firm.  In equilibrium, $\overline q_t = q_t$, but we
must distinguish between $q_t$ and $\overline q_t$ in posing the optimum
problem of a competitive firm.

\subsection{The competitive fringe}

 The representative competitive firm regards $\{p_t\}_{t=0}^\infty$
as an exogenous  stochastic process and chooses
an output plan to
maximize
$$ E_0 \sum_{t=0}^\infty \beta^t \left\{
 p_t q_t - \sigma_t
% .5 c(q_{t+1} - q_t)^2 -.5h q_t^2  - d q_t
 \right\}, \quad \beta \in(0,1) \EQN oli3 $$
subject to $q_0$ given, where  %$c>0, d>0, h>0$ are cost parameters,
$E_t$ is the mathematical expectation based on time
$t$ information.
Let $i_t = q_{t+1} - q_t.$  We
regard $i_t$ as the representative firm's control at $t$.  The
first-order conditions
 for maximizing \Ep{oli3} are
$$  i_t =  E_t  \beta i_{t+1} -c^{-1} \beta h  q_{t+1}
  + c^{-1} \beta  E_t( p_{t+1} -d) \EQN oli4   $$
for $t \geq 0$.
We appeal to the  certainty equivalence principle stated on
page \use{certequiv} to justify working with
a non-stochastic version of \Ep{oli4} formed by dropping
the expectation operator and the random term $\check \epsilon_{t+1}$
from \Ep{oli2}.   We use
a method of Sargent (1979) and Townsend
(1983).\NFootnote{They used this method to compute a rational expectations
competitive equilibrium.  The  key step was
to eliminate price and output by substituting
  from the inverse demand curve and the production function into
the firm's first-order conditions to get a difference equation
in capital.}
  We shift  \Ep{oli1} forward one period, replace conditional
expectations with realized values,  use \Ep{oli1} to  substitute
for $p_{t+1}$ in  \Ep{oli4}, and set $q_t = \overline q_t$ for all
$t\geq 0$ to get
%{\ninepoint
$$ i_t = \beta i_{t+1}  - c^{-1} \beta h \overline q_{t+1}
 + c^{-1} \beta (A_0-d) - c^{-1} \beta    A_1 \overline q_{t+1}
  -  c^{-1} \beta    A_1 Q_{t+1} + c^{-1} \beta    v_{t+1}. \EQN oli5 $$
%}%endninepoint
Given sufficiently stable sequences $\{Q_t, v_t\}$, we could solve \Ep{oli5}
and $i_t = \overline q_{t+1} - \overline q_t$ to
express the competitive fringe's
output sequence  as a function of the (tail of the)
monopolist's output sequence.
The dependence of $i_t$ on future $Q_t$'s opens an avenue for the monopolist to influence current outcomes by its choice now
of its future actions.  
  It is this feature that makes the  monopolist's problem fail  to be
recursive in the natural state variables $\overline q, Q$.   The monopolist
arrives at period $t >0$ facing the  constraint that it must
 confirm the expectations about its time $t$ decision
upon which the competitive fringe   based its decisions at dates
before $t$.

\subsection{The monopolist's problem}

The monopolist views the competitive firm's sequence of Euler equations
as  constraints on its own opportunities.
They are {\it implementability constraints\/} on the
monopolist's choices.
 Including the implementability constraints \Ep{oli5},
we can represent
the constraints
in terms of the transition law impinging on the monopolist:
%{\ninepoint
$$ \eqalign{ \left[\matrix{ 1 & 0 & 0 & 0 & 0 \cr
                  0 & 1 & 0 & 0 & 0 \cr
                  0 & 0 & 1 & 0 & 0 \cr
                  0 & 0 & 0 & 1 & 0 \cr
                  A_0 -d & 1 & - A_1 & - A_1 -h & c \cr }\right]
   \left[\matrix{ 1 \cr v_{t+1} \cr Q_{t+1} \cr \overline
 q_{t+1} \cr i_{t+1} \cr}
    \right]
  & = \left[ \matrix{ 1 & 0 & 0 & 0 & 0 \cr
             0 & \rho & 0 & 0 & 0 \cr
             0 & 0 & 1 & 0 & 0 \cr
             0 & 0 & 0 & 1 & 1 \cr
             0 & 0 & 0 & 0 & {c\over \beta} \cr} \right]
     \left[ \matrix{ 1 \cr v_t \cr Q_t \cr \overline
    q_t \cr i_t \cr} \right] \cr
& + \left[\matrix{ 0 \cr 0 \cr 1 \cr 0 \cr 0 \cr}\right] u_t
   , \cr}   \EQN oli6 $$
%}%endninepoint
where $u_t = Q_{t+1} - Q_t $ is the control of the monopolist.
The last row portrays the implementability constraints \Ep{oli5}.
Represent \Ep{oli6} as
$$ y_{t+1} = A y_t + B u_t .  \EQN oli6a  $$

Although we have entered the competitive fringe's choice variable  $i_t$  as a component
of the ``state''  $y_t$ in the monopolist's transition law  \Ep{oli6a},
$i_t$ is actually  a ``jump''
 variable. Nevertheless, the analysis  in earlier sections of this chapter
implies that the  solution of the large firm's
problem is encoded in the Riccati equation associated with
\Ep{oli6a} as the transition law.  Let's decode it.

To match our general setup, we partition $y_t$ as
$y_t' = \left[\matrix{z_t' &  x_t' \cr} \right]$ where
$z_t' = \left[\matrix{ 1 & v_t & Q_t & \overline q_t \cr}\right]$
and $x_t = i_t$.
 The large firm's problem is
$$
\max_{\{u_t, p_t, Q_{t+1}, \overline q_{t+1}, i_t\}}
 \sum_{t=0}^\infty \beta^t \left\{ p_t Q_t  - {\cal C}_t \right\} $$
subject to  the given initial condition
for $z_0$, equations \Ep{oli1} and \Ep{oli5} and $i_t = \overline q_{t+1} -
\overline q_t$,
 as well as the laws of motion
of the natural state variables $z$.     Notice that the monopolist  in effect chooses the
price sequence, as well as the quantity sequence of the
competitive fringe, albeit subject to the restrictions imposed by
the behavior of consumers, as summarized by the demand curve
\Ep{oli1} and the implementability constraint \Ep{oli5} that
describes the best responses  of the competitive fringe.

By substituting \Ep{oli1} into  the above objective function,
the monopolist's problem can be expressed as
$$
\max_{\{u_t\}}
 \sum_{t=0}^\infty \beta^t
    \left\{ (A_0 - A_1 (\overline q_t + Q_t) + v_t) Q_t - eQ_t - .5gQ_t^2 -
    .5 c u_t^2
 \right\} \EQN oli7  $$
subject to \Ep{oli6a}.
This can be written
$$
\max_{\{u_t\}}
 -  \sum_{t=0}^\infty \beta^t \left\{ y_t' R y_t +   u_t' Q u_t
   \right\} \EQN oli9 $$
subject to \Ep{oli6a}
where
$$  R =  - \left[\matrix{ 0 & 0 & {A_0-e \over 2} & 0 & 0 \cr
                       0 & 0 & {1 \over 2} & 0 & 0 \cr
                       {A_0-e \over 2} & {1 \over 2} & - A_1 -.5g
                   & -{A_1 \over 2} & 0 \cr
                   0 & 0 & -{A_1 \over 2} & 0 & 0 \cr
                  0 & 0 & 0 & 0 & 0 \cr} \right] $$
and $Q= {c \over 2}$.


\subsection{Equilibrium representation}

 We can use \Ep{king11} to  represent the
solution of the monopolist's problem  \Ep{oli9}  in the form:
$$ \left[\matrix{z_{t+1} \cr \mu_{x,t+1}\cr}\right]
   = \left[\matrix{m_{11} & m_{12} \cr
                   m_{21} & m_{22}\cr}\right]
     \left[\matrix{z_t \cr \mu_{x,t} \cr} \right]  \EQN oli11 $$
or
$$ \left[\matrix{z_{t+1} \cr \mu_{x,t+1}\cr}\right]
   = m
     \left[\matrix{z_t \cr \mu_{x,t} \cr} \right] . \EQN oli11 $$
 The monopolist is
constrained to set $\mu_{x,0} \leq 0$, but will find it optimal to
set it to zero.
Recall that $z_t =\left[\matrix{ 1 & v_t & Q_t & \overline q_t \cr}\right]'$.
Thus, \Ep{oli11}  includes the equilibrium law of motion for the quantity
$\overline q_t$
of the competitive fringe.  By construction,  $\overline q_t$ satisfies the Euler
equation of the representative firm in the competitive fringe, as
we elaborate in Appendix \the\chapternum\use{appCblkstack}.
%% TTTTTT
\subsection{Numerical example}
We computed the optimal Stackelberg plan
for parameter settings $A_0, A_1, \rho, C_\epsilon,\hfil\break
  c, d, e, g, h,  \beta $ = $100, 1, .8, .2, 1,  20, 20, .2, .2,
.95$. \NFootnote{These calculations were performed
by the Matlab program {\tt oligopoly5.m}}\mtlb{olipololy5.m}%
%, a modification of Stijn and Tom's earlier
%program with robustness.XXXXX}
For these parameter values
the decision rule  is
$$u_t = (Q_{t+1} - Q_t) =\left[\matrix{ 19.78 & .19 & -.64 & -.15 & -.30 \cr}\right]
\left[ \matrix{z_t \cr \mu_{xt}\cr}\right] \EQN urule1 $$
which can also be represented
as

$$u_t=
    0.44  u_{t-1} +
\left[\matrix{
   19.7827  \cr  0.1885 \cr   -0.6403  \cr  -0.1510 \cr}\right]'  z_t +
\left[\matrix{ -6.9509 \cr   -0.0678 \cr   0.3030  \cr  0.0550 \cr}\right]'
 z_{t-1} . \EQN urule2 $$
Note how in representation \Ep{urule1}
the monopolist's decision for $u_t = Q_{t+1} - Q_t$
feeds back negatively on the implementation
multiplier.\NFootnote{We also computed impulse responses to the demand innovation $\epsilon_t$.
%while the remaining figures show a simulated sample path using
%a Gaussian i.i.d.\ $\epsilon_t$ with mean zero and unit
%variance.  The simulation was initialized with all variables
%except $\mu_x$ set to the nonstochastic
%steady state of \Ep{oli11}; we set $\mu_x$ to zero.
The impulse responses show that a demand innovation
pushes the implementation multiplier down and leads
the monopolist to expand output while the representative
competitive firm contracts output in subsequent periods.  The response
of price to a demand shock innovation is to rise on impact
but then to decrease in subsequent periods in response to the
increase in total supply $\overline q+Q$ engineered by the monopolist.}

 %Note in Figure  4.5 %\Fg{oli30}
%how starting from $0$ the implementation multiplier
%decreases toward its negative steady state value.  The negative
%value of the multiplier reflects the cost to the large firm
%of adhering to its plan.  The time inconsistency
%of the large firm's plan is reflected in the incentive
%the large firm would have to reset the multiplier to zero
%in any period and thereby reinitialize its plan (see Hansen,
%Epple, and Roberds (1985)).   Figure  4.5 %\Fg{oli30}
%and the other sample paths show that the large firm is acting
%to smooth total output $Q+q$, and that it does so by inducing
%a negative contemporaneous covariance between its own output
%and the price.
%%
%%%%%%%%%
%%\midinsert
%%$$ \grafone{oli10.eps,height=2.5in}
%%{{\bf Figure 4.2.}
%%Impulse response of $p, q, Q, \mu_x$ to innovation to demand
%%shock $\epsilon$.} $$
%%\endinsert
%%%%%%%%
%
%\midfigure{oli10f}
%\centerline{\epsfxsize=3truein\epsffile{oli10.eps}}
%\caption{Impulse response of $p, q, Q, \mu_x$ to innovation to demand shock
%$\epsilon$.}
%\infiglist{oli10f}
%\endfigure
%
%%%%%%%%
%%\midinsert
%%$$\grafone{oli60.eps,height=2.5in}{{\bf Figure 4.3.}
%%Impulse response of  $q+Q, w, v$ to $\epsilon$.} $$
%%\endinsert
%%%%%%%%%%%%
%
%\midfigure{oli60f}
%\centerline{\epsfxsize=3truein\epsffile{oli60.eps}}
%\caption{Impulse response of  $q+Q, w, v$ to $\epsilon$.}
%\infiglist{oli60f}
%\endfigure
%
%%%%%%%%%%%%
%%\midinsert
%%$$
%%\grafone{oli20.eps,height=2.5in}{{\bf Figure 4.4.} Sample path of $q+Q, q, Q$.}
%% $$
%%\endinsert
%%%%%%%%%%%%%%%
%
%\midfigure{oli20f}
%\centerline{\epsfxsize=3truein\epsffile{oli20.eps}}
%\caption{Sample path of $q+Q, q, Q$.}
%\infiglist{oli20f}
%\endfigure
%
%%%%%%%%%%%
%% \midinsert
%%$$\grafone{oli30.eps,height=3in}
%% {{\bf Figure 4.5.}  Sample path of $\mu_x, Q, p$.} $$
%% \endinsert
%%%%%%%%%%%%%%%%%%%%%%%%%%%%%
%
%\midfigure{oli30f}
%\centerline{\epsfxsize=3truein\epsffile{oli30.eps}}
%\caption{Sample path of $\mu_x, Q, p$.}
%\infiglist{oli30f}
%\endfigure
%
%%%%%%%%%%%%%%%%%%
%% \midinsert
%%$$
%%\grafone{oli50.eps,height=3in}{{\bf Figure 4.6.} Sample path of $v, Q, q, p$.}
%%$$
%% \endinsert
%%%%%%%%%%%%%%%%%%%
%
%\midfigure{oli50f}
%\centerline{\epsfxsize=3truein\epsffile{oli50.eps}}
%\caption{Sample path of $v, Q, q, p$.}
%\infiglist{oli50f}
%\endfigure

\section{Concluding remarks}

  This chapter is our first encounter with  a class of problems
in which optimal decision rules are history dependent.\NFootnote{For another application of the techniques
in this chapter and how they related to the method recommended by 
Kydland and Prescott (1980), see Evans and Sargent (2013).} We shall
confront many more such problems in chapters \use{socialinsurance},
\use{socialinsurance2},
and \use{credible} and shall see in various contexts how history
dependence can be represented recursively by
appropriately augmenting the natural state variables with
counterparts to our implementability multipliers.  A hint at what
these counterparts are is gleaned by appropriately interpreting
implementability multipliers as derivatives of value functions. In
chapters \use{socialinsurance}, \use{socialinsurance2}, and \use{credible}, we make dynamic
incentive and enforcement problems recursive  by augmenting the
state with continuation values of other decision
makers.\NFootnote{In chapter \use{socialinsurance}, we describe Marcet and Marimon's (1992, 1999) method of
constructing recursive contracts, which  is closely related to the method
that we have presented in this chapter.}
\auth{Marcet, Albert}%
 \auth{Marimon, Ramon}%
  \auth{Kydland, Finn E.} \auth{Prescott, Edward C.}%
  \auth{Evans, David} \auth{Sargent, Thomas J.}%

\appendix{A}{The stabilizing $\mu_t = Py_t$}\label{appAblkstack}%
We verify that the  $P$ associated with the stabilizing $\mu_0 = P
y_0$ satisfies the Riccati equation associated with the Bellman
equation. Substituting $\mu_t = P y_t$ into \Ep{olrp4} and
\Ep{foc1;b} gives
$$\EQNalign{ (I + \beta   B   Q^{-1}   B P) y_{t+1}
   & = A y_t \EQN olrp9;a \cr
    \beta A' P y_{t+1} & = - Ry_t + P y_t. \EQN olrp9;b \cr } $$
A matrix inversion identity implies
$$ (I + \beta   B   Q^{-1}   B' P)^{-1}
  = I - \beta   B (  Q + \beta
    B' P   B)^{-1}   B' P .
  \EQN olrp10  $$
Solving \Ep{olrp9;a} for $y_{t+1}$ gives
$$ y_{t+1} = (A -   B F) y_t \EQN olrp11 $$
where
$$ F = \beta (  Q + \beta   B' P   B)^{-1}   B' P A .\EQN olrp12 $$
Premultiplying \Ep{olrp11} by $ \beta A' P$
gives
$$ \beta A' P y_{t+1} = \beta (A'PA - A' P   B F) y_t. \EQN olrp13 $$
For the right side of \Ep{olrp13} to agree with
the right side of \Ep{olrp9;b} for any initial value of
$y_0$ requires that
$$ P = R + \beta A'P A -\beta^2 A'P   B (  Q +  \beta   B' P
    B)^{-1}   B' P A. \EQN olrp14 $$
Equation \Ep{olrp14} is the algebraic matrix Riccati equation
associated with the optimal linear regulator
for the system $A,   B, Q,   R$.

\appendix{B}{Matrix linear difference equations}\label{appBblkstack}%
This appendix generalizes some calculations from chapter
\use{dplinear} for solving systems of linear difference equations.
Returning to system \Ep{olrp8}, let $L =L^* \beta^{-.5}$ and
transform the system \Ep{olrp8} to
$$ L \left[ \matrix{ y_{t+1}^*  \cr
                \mu_{t+1}^* \cr} \right] =
      N  \left[ \matrix{ y_{t}^*  \cr
           \mu_t^* \cr} \right] ,            \EQN symplec2     $$
where $y_t^* = \beta^{t/2} y_t,  \mu_t^* = \mu_t \beta^{t/2}$.
Now $\lambda L - N$ is a symplectic pencil,\NFootnote{A {\it pencil\/}
$\lambda L - N$ is the family of matrices indexed by the complex variable
$\lambda$.  A pencil is {\it symplectic\/} if $L J L' = N J N'$, where
$J = \left[\matrix{ 0 & - I \cr
                    I & 0 \cr} \right]$. See Anderson, Hansen,
McGratten, and Sargent (1996).}
so that the generalized eigenvalues of
$L, N$ occur in reciprocal pairs: if $\lambda_i$ is
an eigenvalue, then so is $\lambda_i^{-1}$.

We can use Evan Anderson's Matlab program {\tt schurg.m} to
find a stabilizing solution of
system \Ep{symplec2}.\NFootnote{This program is
 available at
% http://www.math.niu.edu/~anderson/
% $<$http://www.unc.edu/\raise-4pt\hbox{\~{}}ewanders$>$.
 $<$http://www.math.niu.edu/\raise-4pt\hbox{\~{}}anderson$>$.}
%{\tt www.unc.edu/\~ewanders}.}
% (This is overkill
%for our problem because $L$ is nonsingular.)
  The program computes the ordered real
generalized Schur decomposition of the matrix pencil.
  Thus, {\tt schurg.m} computes
matrices $\bar L, \bar N, V$ such that $\bar L$ is upper
triangular, $\bar N$ is upper block triangular, and $V$ is the
matrix of right Schur vectors such that for some orthogonal
matrix $W$, the following hold:
$$ \eqalign{ W L V & =  \bar L \cr
         W N V & = \bar N. \cr} \EQN schur $$
Let the stable eigenvalues (those less than $1$)
appear first.  Then the stabilizing solution is
$$ \mu_t^* = P y_t^* \EQN chisoln $$
where
$$ P = V_{21}  V_{11}^{-1},$$
$V_{21}$ is the lower left block of $V$, and
$V_{11}$ is the  upper left block.


If $L$ is nonsingular, we can  represent the solution of the system
as\NFootnote{The solution method
in the text assumes that $L$ is nonsingular
and well conditioned.  If it is not, the following method proposed
by Evan Anderson will work.
We want to solve for a solution of the form
$$ y_{t+1}^* = A_o^{*} y_t^* .$$
Note that with \Ep{chisoln},
$$ L [I; P] y_{t+1}^* = N [I; P] y_t^* $$
The solution $A_o^{*}$ will then  satisfy
$$    L [I; P] A_o^{*} = N [ I;P]. $$
Thus $A^{o*}$ can be computed via the Matlab command
$$ A_o^{*} = (L* [I; P]) \backslash (N* [ I;P]). $$}
$$ \left[ \matrix {y_{t+1}^* \cr \mu_{t+1}^* \cr} \right]
       = L^{-1} N \left[ \matrix{I \cr P \cr} \right] y_t^*. \EQN Zsoln $$
The solution is to be initialized from
\Ep{chisoln}.
We can use the  first half and then the second half
of the rows of this  representation
to deduce the following recursive
 solutions for $y_{t+1}^*$ and $\mu_{t+1}^*$:
$$ \eqalign{ y_{t+1}^* &  = A_o^{*} y_t^*  \cr
             \mu_{t+1}^* & =   \psi^* y_t^*.  \cr } \EQN solnprelim  $$
Now express this solution in terms of the original variables:
$$ \eqalign{ y_{t+1} &  = A_o y_t  \cr
             \mu_{t+1} & =   \psi y_t, \cr } \EQN soln  $$
 where $A_o = A_o^{*}\beta^{-.5}, \psi =
\psi^* \beta^{-.5}$. We also have the representation
$$ \mu_t = P y_t .   \EQN chicontemp $$
The matrix $A_o = A -   B F$, where $F$ is the matrix for
the optimal decision rule.

\appendix{C}{Forecasting formulas}\label{appCblkstack}%
The decision rule for the competitive fringe incorporates
forecasts of future prices from \Ep{oli11} under $m$.
Thus, the representative competitive firm uses equation \Ep{oli11}
to forecast future values of $(Q_t, q_t)$ in order to forecast
$p_t$.
The representative competitive firm's
forecasts
 are generated from the $j$th iterate of \Ep{oli11}:\NFootnote{The
representative
firm  acts as though $(q_t, Q_t)$ were exogenous to it.}
$$ \left[\matrix{z_{t+j} \cr \mu_{x,t+j}\cr}\right]
   = m^j
     \left[\matrix{z_t \cr \mu_{x,t} \cr} \right] . \EQN oli12 $$

%It is important to recognize that the
%representative firm is forecasting
%using $m$ in \Ep{oli2}.
The following calculation verifies that the representative
firm forecasts by iterating  the law of motion
associated with $m$. Write  the Euler equation for $i_t$ \Ep{oli4}
in terms  of a
polynomial
in the lag operator $L$
 and factor it:
$(1 - (\beta^{-1} + (1+c^{-1}h))L + \beta^{-1} L^2) = -(\beta
\lambda)^{-1} L (1 - \beta \lambda L^{-1})(1-\lambda L)$ where
$\lambda \in (0,1)$ and $\lambda =1$ when $h =0$.\NFootnote{See
Sargent (1979 or 1987a) for an account of the method we are using
here.}
  By taking the nonstochastic version
of \Ep{oli4} and solving an unstable root forward and a stable root
backward using the technique of Sargent (1979 or 1987a, chap. IX), we obtain
$$ i_t  =  (\lambda-1)q_t +  c^{-1}   \sum_{j=1}^\infty
   ( \beta \lambda)^j p_{t+j}, \EQN oli4a $$
or
$$ i_t = (\lambda -1) q_t + c^{-1} \sum_{j=1}^\infty
  (  \beta \lambda)^j
     [(A_0-d) - A_1 (Q_{t+j} + q_{t+j}) + v_{t+j}] , \EQN oli4b $$
This can be expressed as
$$ i_t =(\lambda -1) q_t + c^{-1} e_p \beta \lambda m
    (I - \beta \lambda m)^{-1}
\left[\matrix{z_t \cr \mu_{xt}\cr}
  \right]   \EQN oli4c $$
where $e_p = \left[\matrix{ (A_0 -d ) & 1 & - A_1 & -A_1 & 0\cr}\right]$
is a vector that forms $p_t -d$ upon postmultiplication by
$\left[\matrix{z_t \cr \mu_{xt}\cr}
  \right]  $.
It can be verified that
the solution procedure builds in  \Ep{oli4c} as an identity,
so that
\Ep{oli4c} agrees with%\NFootnote{The Matlab program
%{\tt oligopoly2.m} verifies this equality.}
$$ i_t = - P_{22}^{-1} P_{21} z_t + P_{22}^{-1} \mu_{xt}.
   \EQN oli4d $$


%To find a stabilizing subspace, recall from chapter \use{rgames1} some
%of the properties of \Ep{olrp8}.
%The generalized eigenvalues of
%$(L^*,N)$ occur in $\sqrt{\beta}$-symmetric pairs: if $\lambda_i$
%is an eigenvalue, there is another eigenvalue $\lambda_{-i} = {1
%\over \beta \lambda_i}$.
% To find
%the optimum, we must  solve stable roots backward and
%unstable roots forward, which puts $(y_t, \mu_t)$ into
%the stabilizing subspace defined by the Euler equations.
%We can obtain such
%%%%


%\section{Exercises}
\showchaptIDfalse
\showsectIDfalse
\section{Exercises}
\showchaptIDtrue
\showsectIDtrue
\medskip
\noindent{\it Exercise \the\chapternum.1} \quad   There is no uncertainty.
For $t \geq 0$, a  monetary authority sets the growth of the (log)
of money according to
$$ m_{t+1} = m_t + u_t \leqno(1)  $$
subject to the initial condition $m_0>0$ given.  The demand for money
is
  $$  m_t - p_t = - \alpha (p_{t+1} - p_t), \alpha > 0,  \leqno(2)     $$
where $p_t$ is the log of the price level.  Equation (2) can be
interpreted as the Euler equation  of the holders of money.

\medskip
\noindent{\bf a.}  Briefly interpret how equation
(2) makes the demand for real balances vary inversely with
the expected rate of inflation.
Temporarily (only for this part of the exercise) drop
equation (1) and assume instead that $\{m_t\}$ is a given sequence
satisfying $\sum_{t=0}^\infty m_t^2 < + \infty$.
Please solve the difference equation (2) ``forward''
to express $p_t$ as a function of current and future values of $m_s$.
Note how future values of $m$ influence the current price level.

\medskip
At time $0$,  a  monetary authority chooses a possibly
history-dependent strategy for setting $\{u_t\}_{t=0}^\infty$.  (The monetary
authority commits to this strategy.)  The monetary authority orders
sequences $\{m_t, p_t\}_{t=0}^\infty$ according to
$$ - \sum_{t=0}^\infty .95^t \left[  (p_t - \overline p)^2 +
    u_t^2 + .00001 m_t^2  \right]. \leqno(3) $$
Assume that $m_0=10, \alpha=5, \bar p=1$.
\medskip
\noindent{\bf b.} Please briefly interpret  this problem
as one where the monetary authority wants
to stabilize the price level, subject
to costs of adjusting the money supply and some implementability
constraints.    (We include the term $.00001m_t^2$ for purely technical
reasons that you need not discuss.)

\noindent {\bf c.} Please write and run a Matlab program
to find the optimal  sequence
$\{u_t\}_{t=0}^\infty$.
\medskip
\noindent {\bf d.}  Display the optimal decision rule for $u_t$
as a function of $u_{t-1},  m_t, m_{t-1}$.
\medskip
\noindent{\bf e.} Compute the optimal $\{m_t, p_t\}_t$
 sequence for $t=0, \ldots,  10$.

\medskip
\noindent{\it Hint:} The optimal $\{m_t\}$ sequence must satisfy
$ \sum_{t=0}^\infty (.95)^t m_t^2 < +\infty$.
You are free to apply the Matlab program {\tt olrp.m\/}.
% that is available
%from the course web site or from Yongs Shin.

\medskip
\noindent{\it Exercise \the\chapternum.2}  \quad A representative
consumer has quadratic utility functional
$$ \sum_{t=0}^\infty \beta^t \left\{ -.5 (b -c_t)^2 \right\} \leqno(1) $$
where $\beta \in (0,1)$, $b = 30$,  and $c_t$ is time $t$ consumption.
The consumer faces a sequence of budget constraints
$$ c_t + a_{t+1} = (1+r)a_t + y_t - \tau_t \leqno(2) $$
where $a_t $ is the household's holdings of an  asset at the beginning
of $t$, $r >0$ is a constant net interest rate satisfying $\beta (1+r) <1$,
 and
$y_t$ is the consumer's endowment at $t$.  The consumer's plan for
$(c_t, a_{t+1})$  has to obey the boundary condition
$\sum_{t=0}^\infty \beta^t a_t^2 < + \infty$.
Assume that $y_0, a_0$ are given
initial conditions and that
$y_t$ obeys
$$ y_t = \rho y_{t-1}, \quad t \geq 1,  \leqno(3)$$
where $|\rho| <1$.
Assume that $a_0=0$, $y_0=3$, and $\rho=.9$.


At time $0$, a    planner commits to a plan
for taxes $\{\tau_t\}_{t=0}^\infty$.  The planner designs the plan
   to maximize
$$ \sum_{t=0}^\infty \beta^t
\left\{ -.5 (c_t-b)^2 -   \tau_t^2\right\}  \leqno(4) $$
over $\{c_t, \tau_t\}_{t=0}^\infty$ subject
 to the implementability constraints
(2) for $t\geq 0$ and
$$\lambda_t =  \beta (1+r) \lambda_{t+1} \leqno(5) $$
for $t\geq 0$, where $\lambda_t \equiv (b-c_t)$.

\medskip
\noindent{\bf a.}  Argue that (5) is the Euler equation for a consumer
who maximizes (1) subject to (2), taking $\{\tau_t\}$ as a given sequence.
\medskip
\noindent{\bf b.}  Formulate the planner's problem as a Stackelberg problem.
\medskip
\noindent{\bf c.}  For $\beta=.95, b=30, \beta(1+r)=.95$,
  formulate an artificial
optimal linear regulator  problem and use it to solve the Stackelberg problem.
\medskip
\noindent{\bf d.} Give a recursive representation  of the
Stackelberg plan for $\tau_t$.


\eqnotracefalse
